\documentclass[12pt]{article}
\usepackage[T2A]{fontenc}
\usepackage[utf8]{inputenc}
\usepackage[english,russian]{babel}
\usepackage{amsmath, amssymb, amsfonts, amsthm}
\usepackage{hyperref}
\setlength{\parindent}{0pt}
\usepackage{listings}
\usepackage{xcolor}
\usepackage[left=1.5cm,right=1cm,top=1.5cm,bottom=1.5cm]{geometry}


\title{Конспекты по C++. \\\large Семестр 2}
\author{Марк Тюков, Иван Кудрявцев}
\date{Весна 2020 года}

\pagestyle{headings}

\lstset{ 
  backgroundcolor=\color{white},   % choose the background color; you must add \usepackage{color} or \usepackage{xcolor}; should come as last argument
  basicstyle=\footnotesize,        % the size of the fonts that are used for the code
  breakatwhitespace=false,         % sets if automatic breaks should only happen at whitespace
  breaklines=true,                 % sets automatic line breaking
  captionpos=b,                    % sets the caption-position to bottom
  commentstyle=\color{black},      % comment style
  deletekeywords={...},            % if you want to delete keywords from the given language
  escapeinside={\%*}{*)},          % if you want to add LaTeX within your code
  extendedchars=\true,             % lets you use non-ASCII characters; for 8-bits encodings only, does not work with UTF-8
  frame=single,	                   % adds a frame around the code
  keepspaces=true,                 % keeps spaces in text, useful for keeping indentation of code (possibly needs columns=flexible)
  keywordstyle=\color{blue},       % keyword style
  language=C++,                    % the language of the code
  morekeywords={*,...},            % if you want to add more keywords to the set
  numbers=left,                    % where to put the line-numbers; possible values are (none, left, right)
  numbersep=8pt,                   % how far the line-numbers are from the code
  numberstyle=\color{gray},        % the style that is used for the line-numbers
  rulecolor=\color{gray},          % if not set, the frame-color may be changed on line-breaks within not-black text (e.g. comments (green here))
  showspaces=false,                % show spaces everywhere adding particular underscores; it overrides 'showstringspaces'
  showstringspaces=false,          % underline spaces within strings only
  showtabs=false,                  % show tabs within strings adding particular underscores
  stepnumber=1,                    % the step between two line-numbers. If it's 1, each line will be numbered
  stringstyle=\color{violet},      % string literal style
  tabsize=2,	                   % sets default tabsize to 2 spaces
  texcl=true,
  title=\lstname                   % show the filename of files included with \lstinputlisting; also try caption instead of title
}

\newcommand{\ps}{\textbf{P.S.: }}

\newcommand{\cppEleven}{\textbf{C++11}}

%%%%%%%%%%%%%%%%%%%%%%%%%%%%%%%%%%%%%%

\begin{document}

%%
\maketitle
\tableofcontents
\newpage
%%

\section{Введение в язык}
\subsection{Инструкции}

\subsubsection{Declarations (объявления)}

\textbf{Переменные}
\begin{lstlisting}[language=C++]
type id [= value];
\end{lstlisting}

\textbf{Примеры}

[unsigned] int/long long/char/float/double/long double/bool

\ps size\_t $\equiv$ unsigned long long

\

\textbf{Функции, структуры}

\begin{lstlisting}
void f(int x, double y){}
struct S{};
\end{lstlisting}

\ps using vi = std::vector<int>;

\paragraph{Declaration vs definition}

!!! One definition rule (ODR)

\subsubsection{Expressions}

\textbf{Базовые операторы} 

\begin{enumerate}
	\item Арифметические операторы (+, -, $\ast$, /, $\%$) 
	\item Побитовые операторы ($\&$, |, $\wedge$, $\sim$, $\ll$, $\gg$)
	\item Логические операторы ($\&\&$, ||, !)
	\item Операторы сравнения (==, <, >, <=, >=)
%% ВСТАВИТЬ ПРО Lvalue И Rvalue
	\item Assignments ($=$, $+=$, $-=$, $\ast=$, $/=$, $\%=$, $\&=$, $|=$, $\space\wedge$=, $\ll=$, $\gg=$)
	\item Инкремент и декремент ($++x$, $x++$)
	\item sizeof() - возвращает число, которое нужно для хранения входных данных (в байтах). Он не сохраняет результат операций. Например, $sizeof(x++)$ не изменит $x$
	\item Тернарный оператор "... ? ... : ..."
	\item Запятая ---- выполняет левую часть, потом правую, возвращает правую. Она гарантирует, что левая часть закончит выполнение до того, как начнет выполняться правая. 
\end{enumerate} 

\subsubsection{Control statements}

\begin{enumerate}
	\item if (statement) {} else {}
	\item while(statement){}
	\item for (declaration or expression; bool expression; expression){}
\end{enumerate}

\section{Ошибки}

\subsection[Compilation Error]{Ошибки компиляции (Compilation Error)} 

\subsubsection{Лексические ошибки}
Неизвестный символ

\subsubsection{Синтаксические ошибки}
\texttt{if = 5)}

\subsubsection{Семантические ошибки}
\colorbox{red}{...}

Семантическая ошибка (или ещё «смысловая») возникает, когда код синтаксически правильный, но выполняет не то, что нужно программисту.

\subsection[Runtime Error]{Ошибки выполнения RE - Runtime Error}

\subsubsection[Segmentation Fault]{Segmentation Fault - обращаемся к чужой памяти} 
\begin{enumerate}
	\item \colorbox{red}{заканчивается стек рекурсии}
\end{enumerate}

\subsection{Undefined Behaviour} 

Когда пишем что-то такое, на что компилятор в стандарте не имеет четкой инструкции.

\paragraph{Пример:} обращение к массиву по несуществующему индексу. 
\begin{lstlisting}
int a[10];
a[100]; 
\end{lstlisting}

В таком случае можно получить как RE, так и UB 

\textbf{Если в программе случился UB, то не гарантируется ничего}! 

\

\colorbox{red}{(1 != 0) будет true}

\subsection[Linking Error]{Linking error - ошибки линковщика}

После компиляции, например, что-то не объявлено.

\

\subsection{Насколько плохи все эти ошибки?} 

CE - компилятор наш друг

RE - \underline{ПЛОХО!} Сервер может упасть/может случиться что-то плохое во ВРЕМЯ ВЫПОЛНЕНИЯ

UB - \underline{УЖАСНО!!!} Не совершайте преступление, не делайте UB!

\textit{UB и RE компилятор не блокирует зачастую при компиляции.}

\section{Указатели. Массивы}

\subsection{Pointers, arrays, functions, etc}

\subsubsection{Pointers}

\begin{lstlisting}
int x {
	int y; // выделение памяти
} // удаление из памяти ( на самом деле потеря адреса, что происходит с данными по тому адресу - неизвестно)

type* p; 
*p; // унарная звездочка разыменовывания.

type* p = &x; // кладем в p адрес x
p + 1; p - 1;
\end{lstlisting}

\subsubsection{Операции с указателями}
\begin{enumerate}
\item Разыменование;
\item Сложение с числами;
\item Разность указателей. 
\end{enumerate}

\textbf{void*} - указатель на несуществующую область памяти. Такие указатели нельзя увеличивать и вычитать друг из друга.

\paragraph{Но!} Указатели \underline{можно преобразовывать}. \textit{nullptr} - константный указатель (типа \textit{nullptr\_t}) - аналог нуля для указателей

\

\textbf{P.S.:}
\begin{lstlisting}
*nullptr  // UB :)
\end{lstlisting}

\subsection{Arrays}

type a[10] - выделение памяти на стэке для 10 элементов

\subsubsection{Операции над массивами:} 
\begin{enumerate}
	\item *(a+i) --- взятие адреса, где начинается массив, прибавление к нему числа i и разыменование.
	\item Array-To-Pointer conversion 

	\texttt{type* b = a;}
	
	\item \texttt{sizeof(a)} --- размер массива * \texttt{sizeof(тип)}

	\texttt{sizeof(type*)} $\not=$ \texttt{sizeof(a);}
\end{enumerate}

\subsection{Functions}

\textbf{Сигнатура} - набор типов принимаемых аргументов.

Можно объявить несколько функций с разными сигнатурами
\begin{lstlisting}
type f(int);
type f(double);
type f(int, int);
\end{lstlisting}

Эти функции могут возвращать данные разных типов

Компилятор при вызове таких функция совершает разрешение
 перегрузки \textbf{(overloading resolution)}, то есть принимает решение о выборе версии функции

\subsubsection{Overloading resolution}
\begin{enumerate}
	\item В точности такой тип;
	\item Преобразование в int;
	\item Если не получается однозначно выбрать, будет ошибка компиляции \textbf{Ambiguous Сall}.
\end{enumerate}

Пример \text{f(float)} вызываем, когда есть только от double и от int. 

\ps Читать в стандарте!!!

\subsubsection{Указатель на функцию}

\begin{lstlisting}
int f(double);
int (*)(double) pf = f;

type f(){
}
\end{lstlisting}

\ps Запятая при указании аргументов --- устойчивая конструкция для аргументов, а не \textbf{expression}. 

\texttt{int a = 5;} 

Здесь знак равно --- это не оператор присваивания, а устойчивая конструкция инициализации!

\subsubsection{Аргументы по умолчанию}

Аргументы по умолчанию функций указываются \underline{в конце}. 

\texttt{f(double x, int n = 10)}

\subsubsection{Inline}

Непосредственная вставка кода в указанное место при компиляции. 

\texttt{inline} --- лишь рекомендации компилятору (компилятор решает сам, он умный)


\section[Память]{Статическая и динамическая память}

\subsection{Статическая память}

\subsubsection{Static Variables}

\textbf{Свойства}
\begin{enumerate}
	\item Один раз заводятся;
	\item Размер вычисляет компилятор до запуска программы;
	\item Инициализируются один раз;
	\item Значение при разных вызовах функции сохраняются. 
\end{enumerate}

\subsection[Динамическая память]{Динамическая память - выдается по требованию}

\subsubsection{Операторы}

\texttt{new, delete}

\subsubsection{Переменная}

\begin{lstlisting}
type* p = new type(); // запрашиваем память	
\end{lstlisting}

Потом нужно \underline{освободить} память

\begin{lstlisting}
delete p;
\end{lstlisting}

\subsubsection{Массив}

\begin{lstlisting}
type* p = new type[n]; // запрашиваем память
delete[] p;
\end{lstlisting}

\ps delete и new - expressions

\subsubsection{Некорректное использование delete}

\begin{enumerate}
	\item delete не на тот указатель - UB
	\item delete без [] для массивов - UB
	\item Если не освобождать память возможен \textbf{Memory Leak}
	\item Если дважды удалить, то будет RE (Segmentation Fault)
\end{enumerate}

\ps \colorbox{yellow}{\texttt{delete p, pp;}} --- это \textbf{expression}. Парсится по запятой. Выполнится \texttt{delete p}, потом \texttt{pp} (просто обращение). То есть \texttt{pp} не удалится :(

\section{Ссылки}

\subsection{Создание ссылки}

\texttt{type x;}

\texttt{type\& y = x;} // новое название переменной, y - \textbf{ссылка} на x.

\

Не заработает:
\begin{lstlisting}
void swap(int x /* создаем локальную КОПИЮ икса */, int y){
  int t = x;
  x = y;
  y = t;
} 
\end{lstlisting}

\begin{lstlisting}
type x;
type y = x; // Создаем ссылку, но НЕ в С++. В Java - да 
\end{lstlisting}
\ps Java $\equiv\heartsuit$

\

\begin{lstlisting}
type x;
type& y = x; // новое название переменной. y - ссылка на x. Всё, что делается с y делается и с x 
\end{lstlisting}

Отныне нет способа отличить y от x. Отныне и навсегда:

\textit{``Я поступил на физтех'' $\equiv$ ``Я поступил в МФТИ''}

Правильная реализация swap:

\begin{lstlisting}
void swap(int& x, int& y) {
	int t = x;
	x = y;
	y= t;
}
\end{lstlisting}

\subsection{Операции над ссылками:} (В основном проблемы :))

\begin{enumerate}
	\item Можно всё то, что можно делать с x. 
	\item Нельзя не инициализировать. 
	\item Нельзя делать ссылки на \texttt{rvalue}:
\begin{lstlisting}
int& x = 5;
\end{lstlisting} 
\

	\item Можно:
\begin{lstlisting}
	int x;
	int& f() {
		return x; // x -- глобальный
	}
\end{lstlisting}
	\item Если x в предыдущем примере локальный, то так нельзя, будет битая ссылка (\textbf{Dangling Reference})
\end{enumerate}

\section{Константы}

\subsection{Объявление}

\texttt{const int x = 3;}

Это такой тип, к которому применены только константные операции

\subsection{Что можно и нельзя}

\begin{enumerate}
	\item Необходимо инициализировать сразу при объявлении!!!
	\item Можно передавать не константную версию туда, где нужна константная
\end{enumerate}

\subsection{Константный указатель}

\texttt{const int* p = new int} // указатель на константный инт: \texttt{*p = 1;} -- нельзя; можно \texttt{p++}

\texttt{int* const p = new int} // искомый константный указатель

\subsection{Константная ссылка}

Нельзя \texttt{int$\&$ const x = 1;} // это какая-то фигня. Не надо так :(

Можно заводить ссылки на константные переменные, но не стоит делать константные ссылки (будет либо ошибка, либо не будет иметь смысла)

\

\begin{lstlisting}
int x = 1;
const int& y = x;
\end{lstlisting}

Выше мы можем менять x, но не через y. 

\paragraph{Продление жизни ссылок}\texttt{const int$\&$ x = 5} // можно инициализировать \texttt{rvalue} 

Такая ссылка не умрет, пока не закончиться локальная видимость переменной.\footnote{\text{https://habr.com/ru/post/186790/}} 

\section{Приведение типов}

\subsection{static\_cast}

\begin{lstlisting}
static_cast<type> (expression);
\end{lstlisting}

Если преобразование не однозначно, то будет CE (неоднозначный каст). 

Будет CE, если нет подходящего преобразования. 

Название static из-за того, что всё делается на уровне компиляции. 

Не знаешь, какое приведение типов тебе нужно --- тебе нужен \texttt{static\_cast} 

\subsection{Три запрещенных заклинания:}

\paragraph{Первое заклинание:} \texttt{reinterpret\_cast<>(...)}

Берёт объект как байты в памяти и начинает считать, что это другой тип. 

\

Можно \texttt{reinterpret\_cast} указателей:

\texttt{reinterpret\_cast<type\&>(...);}

\

\texttt{reinterpret\_cast} ссылки:

\texttt{type y = reinterpret\_cast<type\&>(x);}

\paragraph{Второе заклинание:} \texttt{const\_cast<>(...)}

Означает ``перестань считать константу константой'' (которое неявно запрещено) и наоборот. Вообще, это UB. 

\textit{Пример:}

\begin{lstlisting}
int& z = const_cast<int&>(y);	
\end{lstlisting}

Теоретически, это нужно, когда есть две функции:

\begin{lstlisting}
f(int f)
f(const int&)
\end{lstlisting}

Если по какой-то причине захотим запустить f() от инт'а по пути константы (если f для них работает совершенно по разному), то нужно будет привести \texttt{int} к \texttt{const int}. 

\texttt{const\_cast} -- это угнетение компилятора.

\paragraph{Третье заклинание} C-style cast

Пытается сделать всё, чтобы получилось. Так что мы даже не узнаем, что сделалось

\begin{lstlisting}
(type)(expression)	
\end{lstlisting}

\section{Введение в ООП}

\subsection{Что это такое?}

Код состоит из объявления разных объектов некоторых типов и expressions с этими объектами.   

\subsection{Три волшебных слова в ООП}

Инкапсуляция, наследование, полиморфизм --- основные принципы, на которых основано ООП. Дальше мы подробнее изучим каждое слово, а также увидим связь одного с другим. В частности, по мере изучения нового слова могут быть сделаны уточнения и нововведения в предыдущую тему. 

\section{Инкапсуляция}

Точное определение инкапсуляции дать сложно. Ниже приведены два определения на не совсем формальном языке, дабы понять суть данного понятия. 

\textbf{Инкапсуляция} - оборачивание в (защитную) оболочку внутренней реализации с помощью ограничивающего интерфейса. 

\textbf{Инкапсуляция} - совместное хранение полей и методов (но ограниченный доступ к ним извне).  

\subsection[Классы и структуры]{Объявление классов и структур} (авторы ленивые, поэтому описание некоторого базового синтаксиса может быть пропущено)

\begin{lstlisting}
class C {
  /* тело класса */
};

struct S {
  /* тело структуры */
};
\end{lstlisting}

В теле --- методы, поля и т.д.

\subsection{Модификаторы доступа}

\begin{lstlisting}
class C {
 private: // может быть опущено
 public: // дальше всё public, пока не встретится модификатор доступа
 ...
 protected:
 ...
};
\end{lstlisting}
 
Для структуры всё то же самое, только изначально всё публично, а не приватное. 

\subsection{Оператор ``стрелочка''}

\texttt{(*p).f();} $\equiv$ \texttt{p->f();}

\subsection{Указатель this}

\colorbox{red}{...}

\subsection{Конструкторы}

Конструктор нужен для инициализации объектов некоторого класса. 

\begin{lstlisting}
	C x = ... ;
	// варианты инициализации
	C x(...);
	C x{...}; 
	C x = C(...);
	C x = {...}; // так еще можно делать в структуре без конструктора (если все поля паблик) -- агрегатная инициализация. 
\end{lstlisting}


Дальше в какой-то степени будем реализовывать класс строк

\begin{lstlisting}
  class String {
   public:
     
    String() { // конструктор
      str = new char[16]; // начальный размер
      size = 0;
    } 
    
    String(size_t n) {
      ...
    }
     
   private:
    char* str;
    size_t size;  	
  };
\end{lstlisting}

Конструктор по умолчанию -- такое компилятор может сделать сам (он это делает, если мы этого не сделали), но он будет инициализировать по умолчанию все поля (и это в большинстве случаев плохо). 

Первое правило -- компилятор сам создает конструктор, если мы его не определили и не определили никакой другой конструктор. Но можно явно попросить его сгенерировать такой. 

\begin{lstlisting}
	String() = default; // начиная с C++11
\end{lstlisting}

Конструкторы нужны в тех случаях, когда надо инициализировать не тривиально. 

Можно поле создавать так (начиная с \cppEleven):

\begin{lstlisting}
	size_t size = 0; // это будет дефолтная инициализация. 
\end{lstlisting}

Можно сделать делигацию конструктора. Сначала выполнится один конструктор, потом другой. 

\begin{lstlisting}
	String(...) : String(...) {
	  ... // дополнительный код
	}
\end{lstlisting}

Если структура состоит только из 

\subsection{Деструкторы}

При запуске деструктора:

\begin{enumerate}
	\item Удаляем нетривиальные объекты (у которых выделена память оператором \textit{new});
	\item Закрытие потоков;
	\item Освобождение ресурсов; 
\end{enumerate}


\ps Всегда надо подчищать за собой!

Деструктор \textbf{нельзя} сделать приватный. 

Деструктор можно вызывать явно (но в крайних и очень редких случаях). 

\begin{lstlisting}
	~String() {
	  delete[] str;
	}
\end{lstlisting}

Все, что нужно делать в деструкторе -- нетривиальные действия. Все остальные поля уничтожаются сами после выхода из деструктора. 

\subsection{Копирование, копирующее присваивание и правило трех}

Для большинства объектов хочется уметь делать копии. 

\begin{lstlisting}
	S s;
	S s1 = s; // если конструктора копирования нет, компилятор его автоматически создает, просто копируя все поля.  
\end{lstlisting}

Это плохо с нетривиальными полями (ссылки, указатели и т.д.) -- может быть RE. 

\paragraph{Инициализация конструктора копирования} Важно делать \texttt{const String\& s}

\begin{lstlisting}
  String(const String& s) {
    str = new char[s.size];
    for (...) {
       ...
    }	
  };
\end{lstlisting}

Аналогично обычным конструкторам можно написать \texttt{= default} 

Чтобы запретить копирование:

\begin{enumerate}
	\item Сделать приватным
	\item \texttt{String(const String\& s) = delete;} начиная с \cppEleven
\end{enumerate}

Если хотим заменить уже существующий объект на другой, то нужно удалить старый объект и положить туда новый. 

Тривиальный оператор присваивания генерируется автоматически.

Он должен возвращать результат присваивания (того же типа). Хотим возвращать \textit{lvalue}. Возвращаем неконстантную ссылку, чтобы можно было ей присваивать  

\

Если хотим написать 

\begin{lstlisting}
	String s;
	String s1;
	s1 = s;
\end{lstlisting}

то здесь будет вызываться оператор присваивания. 

\begin{lstlisting}
  String& operator =(const String& s) {
    // стоит проверять на присваивание самому себе
    if (this == &s) return *this;
    delete[] str;
    ...
    return *this;
  }
\end{lstlisting}
 
 Оператор присваивания тоже можно писать через \texttt{= default}
 
 \paragraph{Rule of three}  Если в нашем классе потребовалось реализовать что-то одно из нетривиальных конструктора, деструктора или оператора присваивания, то потребуются и все три. 
 
 \subsection{Константные методы}
 
 Такие методы, что их можно выполнять над константными переменными. Надо писать слово \texttt{const}, когда метод ничего не меняет.
 
\begin{lstlisting}
 void f(...) const {
   ...
 }
\end{lstlisting} 
 
 Если мы хотим завести счетчик, сколько раз метод был вызван, а метод константный. В таком случае, если слово ``anticonst'' -- \texttt{mutable}
 
 То есть счетчик будет реализован:
 
\begin{lstlisting}
  mutable int counter;
\end{lstlisting}
 
\begin{lstlisting}
  char& operator [](size_t n);
  const char& operator[](size_t n) const;
\end{lstlisting}
 
 \subsection{Списки инициализации в конструкторах}
 
\begin{lstlisting}
struct S {
	int& x;
	const int y;
	
	S(int& x, int y) { // когда вошли в эту область видимости, поля должны быть уже проинициализированы, а ссылку невозможно проиниализировать так
	}
	
	// Вместо этого так:
	
	S(int& x, int y) : x(x), y(y) {} // здесь инициализация будет до входа в конструктор
};
\end{lstlisting}

Списки инициализации сохраняют нам одно копирование.
 
\subsection{Friends}

Иногда захочется, чтобы приватное поле было доступно. 

\begin{lstlisting}
  friend void f(int);
  friend class C;
\end{lstlisting}
 
 
\begin{lstlisting}
  friend istream& operator >> (istream& in, S& x)
\end{lstlisting}
 
\subsection{Explicit}

Если у нас большой код, то велика возможность что-то пропустить и получить неявное преобразование там, где его не должно быть. Для этого можно запретить неявную конвертацию. 

\begin{lstlisting}
	explicit String(size_t n); // можно вызывать только явно
\end{lstlisting}



Операторы преобразования тоже могут быть \textit{explicit} (c \cppEleven )
\begin{lstlisting}
	explicit operator int() { // оператор преобразования к инту
	  ...
	  return x; // x типа int 
	}
\end{lstlisting}

\subsection{Contextual conversial}

Это конверсия в буль в ифах, форах, вайлах (под условиях). Такая конверсия игнорирует explicit (потому что не является неявных преобразованием)

\subsection{Статические поля и методы}
Это те поля и методы, которые относятся не к конкретному объекту, а ко всему классу в целом. 

\begin{itemize}
	\item Память на них выделяется при компиляции (в статической области).
	\item Из статических методов есть доступ только к статическим полям. 
	\item Если статический метод публичный, то для вызова его извне класса надо писать: 
\end{itemize} 

\begin{lstlisting}
	class C {
 	public:
		static void method();
	}

	int main() {
		C object;
		object.method() // неправильно
	
		C::method() // правильно
	}
\end{lstlisting}
 

\subsection[Pointers to members]{Pointers to members -- указатель на член класса}

\begin{lstlisting}
	int S::* p = &S::*x; // для поля
	int(/*параметры*/) (S::*) ... // для метода 
	
	S s;
	s.*p // вернет ссылку на x // здесь .* - отдельный оператор
\end{lstlisting}

Пример: есть ориентированный граф и мы можем делать обход либо по обычному, либо по инвертированным ребрам. Хотим написать (\underline{одну}) функцию, которая будет делать обход (как обычный, так и инвертированный). В зависимости от того, какой обход требуется, нужно завести указатель на начало и указатель на конец, а в обходе вызываем от указателя. 

\section[Inheritance (наследование)]{Inheritance -- второй принцип ООП}
Некоторые типы могут быть ``подтипами'' других. Производные типы содержат все поля и методы родителей, а также и некоторые свои. 

\textbf{\textit{Синтаксис:}}
\begin{lstlisting}
class Derived : public /*private, protected*/ Base {
	...
}
\end{lstlisting}

\subsection{Модификатор PROTECTED}

Будет доступен членам, друзьям, детям (наследникам)

Стоит обращать внимание на тип, который используется (структура или класс)

\subsection{Размещение в памяти объектов наследников}

\begin{lstlisting}
	struct Base {
	  int a;
	  
	  Base(int a){}
	  
	};
	
	struct Derived : public Base{
	  int a;
	  int b;
	};
\end{lstlisting}

sizeof(Derived) даст 3: a, a, b

\

\begin{lstlisting}
	Derived d;
	d.a; // поле Derived
	d.Base::a; // поле Base
\end{lstlisting}

То есть при создании наследника \underline{всегда} создается родитель (со всеми полями и т.п.), а также сам класс, со всеми его полями. Также при удалении: сначала сам класс, потом родитель. 

\begin{lstlisting}
	Derived(int a, int b, int c) : Base(a), a(b), b(c) {
		...
	}
\end{lstlisting}

Циклическая объявление -- ошибка компиляции. 

\ps Когда пишем деструктор -- не нужно удалять Base !!!

\subsection{Поиск имён при наследовании}

\begin{lstlisting}
	struct Granny {
		int x;
		void f();
	}
	struct Mom : private Granny {
		int d;
		void f(int y);
	}
	struct Son : public Mom {
		int e;
		void f(double y)
	}
 
 	Son s;
	s.f(1); // Тут произойдёт неявный каст в double
\end{lstlisting}

Другие сигнатуры функций будут не видны (\textit{invisible}). Другие будут затменены сигнатурой из Son. 

\begin{center}
visible $\not=$ accessible
\end{center}

Видимые -- те, которые находит поиск имен. 
Доступные -- те, к которым есть доступ по модификаторам доступа при наследовании. 

А если сделать \texttt{private void f(double y)} внутри Son, то будет CE

Решение: \textbf{Qualife id}

\begin{lstlisting}
	s.Mom::f(1);
\end{lstlisting}

\ps Поиск имён происходит всегда до проверки доступа!

\begin{lstlisting}
	s.f(); // Ошибка компиляции, т.к. такая функция invisible или т.к. она private
	s.Mom::f() // то же самое
	s.Granny::f() // CE, т.к. имя Granny inaccessible
\end{lstlisting}

\begin{lstlisting}
	...
	
	class Son : public Mom {
	public void f(double) {
		Granny g; // не сработает
	}
	public void f(double) {
		::Granny g; // сработает
	} 
}
\end{lstlisting}

Разрешим сыну общаться с бабушкой 

\begin{lstlisting}
	...

	class Mom : private Granny{
		friend class Son; // разрешаем сыну общаться с бабушкой
	}
	
	...
	s.Granny::f() // по-прежнему нельзя
\end{lstlisting}
 
\subsection{Multiple inheritance}

\begin{center}
	\underline{ЭТО ПЛОХО. УЖАС. НЕ НАДО ТАК.}
\end{center} 

Почему? Из-за \textbf{проблемы ромбовидного наследования (diamond problem)}. Рассмотрим геометрические фигуры. 

Имеем систему (стрелка ведет от сына к родителю)

\begin{center}
	Square $\rightarrow$ Rectangle; Square $\rightarrow$ Rombus; Rectangle $\rightarrow$ Parallelogram; Rhombus $\rightarrow$ Parallelogram; 
\end{center}

Пусть в каждом лежит по инту. Тогда при создании квадрата создастся \underline{два} инта. 

\textbf{\textit{Синтаксис:}}

\begin{lstlisting}
	class Square : public Rhombus, public Rectangle {};
\end{lstlisting}

При множественном наследование (если оно нам точно нужно), то нужно либо явно вызывать нужный метод (или обращаться к нужному полю), или следить, чтобы в предках не было  методов с одинаковыми сигнатурами. 

Ещё \underline{плохой} пример:

\begin{center}
	Son$\rightarrow$Mom; Mom$\rightarrow$Granny; Son$\rightarrow$Granny
\end{center}

Можно получить \textbf{inaccessible base class}, если захотим обратиться к чему-то из Granny. Из-за структуры у сына будет две бабушки, поэтому не понятно к какому имени мы обращаемся (то есть к полям бабушки). 

\subsection{Virtual наследование}

\begin{lstlisting}
	class class_name1 : public virtual class_name {}
\end{lstlisting}

Когда хотим предотвратить создание объектов в множественном числе используется модификатор \texttt{virtual}

\

На самом деле при таком наследовании размер нового объекта будет даже больше!
\texttt{int x} будет в единственном экземпляре, но будут указатели. 
Также нарушается порядок расположения в памяти: объект может лежать в памяти разрывным куском. 

\colorbox{red}{Вставить рисунки}

На самом деле указатели будут указывать на некое специальное место, где компилятор создает хранилище информации для файлов такого типа

\subsubsection{Private virtual}

При приватном наследовании при попытке обращения к \texttt{x} будет ошибка уже из-за приватности

\colorbox{red}{...}

\subsubsection{Очередное запрещённое заклинание}
$[[[G]M][G]S]$
Делаем 
\begin{lstlisting}
Granny granny = reinterpret_cast<Granny>(mom)
\end{lstlisting}
И тогда можем получить доступ к объекту G (самому левому)

\subsection{Приведение типов при наследовании}

\begin{center}
	Derived \{int y, void f()\}$\rightarrow$Base \{int y, void f()\}
\end{center}

\begin{lstlisting}
	class Base {
		int x;
	}
	
	class Derived : public Base {
		int y;
	}

	Derived d;
	Base* bp = &d;
	Base& b = d;
	b.f() // из Base
	d.x++ // это изменит x и b, так как это разные имена одного и того же 
	Derived* dp = &b; // CE, потому что Derived потомок Base; чтобы так писать нужен явный каст
	Derived& dd = b; // CE аналогично предыдущему
	
	
	Base bb = d; // \textbf{slicing}
\end{lstlisting}

\subsubsection{static\_cast}

\begin{lstlisting}
	Derived& dd = static_cast<Derived&> b; // явный down cast (наследование вниз) (вверх делать тоже можно)
\end{lstlisting}

\ps Такое преобразование очень опасно, так как компилятор не может проверить данный процесс. Если в не лежал Derived, то получим UB. 


Статик каст позволяет перестраховаться от нарушения наследования и нарушения приватности и очень помогает при множественном наследовании. 

\textbf{Пример:}

\begin{center}
	Son$\rightarrow$Mother; S$\rightarrow$Father; Mother$\rightarrow$Granny; Father$\rightarrow$Granny
\end{center}

\begin{lstlisting}
	static_cast<Granny&>(s);
\end{lstlisting}

\subsubsection{reinterpret\_cast}

Данный каст будет игнорировать приватность и все прочее. 

\colorbox{red}{Множественное наследование?}
\begin{lstlisting}
Son s;
Father * fp = &s;
Father & f = s; //Происходит правильный сдвиг указателей
Father f = static_cast<Son> // Правильный сдвиг
Father f = reinterpret_cast<Son> // НЕправильный сдвиг
\end{lstlisting}

\subsection{Виртуальные функции}

Допустим, у нас есть координатная плоскость с двумя кругами: оба радиусом 1, в центре 1 и -1. Если мы опием квадраты вокруг этих кругов, то равны ли эти квадраты?

Другой пример. $\sin^2{x}$ и $f^2(x)$ Первое мы скорее поймем как произведение синусов, а второе --- как композицию одинаковой функции. То есть для более узкого круга вещей (понятий) мы определяем одни и те же символы и формулы по разному. 

Виртуальными функциями называются такие, у которых применяются более частные версии, даже если к объекту обратились как для общего вида. 

\begin{lstlisting}
	class Base {
		void f();
		virtual void g();	
	};
	
	class Derived : public Base {
		void f();
		virtual void g();
	};
	
	...
	Derived d;
	Base& b = d;
	b.f(); // вызывается из Base
	b.g(); // вызывается из Derived
\end{lstlisting}

\ps виртуальные функции замедляют компилятор и время выполнения, потому что компилятору надо понимать что в данном случае мы хотим сделать. 


% 1



% 2



% 3



% 4



% 5



% 6



% 7



% 8



% 9



% 10



\end{document}


























